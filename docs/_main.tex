% Options for packages loaded elsewhere
\PassOptionsToPackage{unicode}{hyperref}
\PassOptionsToPackage{hyphens}{url}
%
\documentclass[
]{book}
\usepackage{amsmath,amssymb}
\usepackage{iftex}
\ifPDFTeX
  \usepackage[T1]{fontenc}
  \usepackage[utf8]{inputenc}
  \usepackage{textcomp} % provide euro and other symbols
\else % if luatex or xetex
  \usepackage{unicode-math} % this also loads fontspec
  \defaultfontfeatures{Scale=MatchLowercase}
  \defaultfontfeatures[\rmfamily]{Ligatures=TeX,Scale=1}
\fi
\usepackage{lmodern}
\ifPDFTeX\else
  % xetex/luatex font selection
\fi
% Use upquote if available, for straight quotes in verbatim environments
\IfFileExists{upquote.sty}{\usepackage{upquote}}{}
\IfFileExists{microtype.sty}{% use microtype if available
  \usepackage[]{microtype}
  \UseMicrotypeSet[protrusion]{basicmath} % disable protrusion for tt fonts
}{}
\makeatletter
\@ifundefined{KOMAClassName}{% if non-KOMA class
  \IfFileExists{parskip.sty}{%
    \usepackage{parskip}
  }{% else
    \setlength{\parindent}{0pt}
    \setlength{\parskip}{6pt plus 2pt minus 1pt}}
}{% if KOMA class
  \KOMAoptions{parskip=half}}
\makeatother
\usepackage{xcolor}
\usepackage{longtable,booktabs,array}
\usepackage{calc} % for calculating minipage widths
% Correct order of tables after \paragraph or \subparagraph
\usepackage{etoolbox}
\makeatletter
\patchcmd\longtable{\par}{\if@noskipsec\mbox{}\fi\par}{}{}
\makeatother
% Allow footnotes in longtable head/foot
\IfFileExists{footnotehyper.sty}{\usepackage{footnotehyper}}{\usepackage{footnote}}
\makesavenoteenv{longtable}
\usepackage{graphicx}
\makeatletter
\newsavebox\pandoc@box
\newcommand*\pandocbounded[1]{% scales image to fit in text height/width
  \sbox\pandoc@box{#1}%
  \Gscale@div\@tempa{\textheight}{\dimexpr\ht\pandoc@box+\dp\pandoc@box\relax}%
  \Gscale@div\@tempb{\linewidth}{\wd\pandoc@box}%
  \ifdim\@tempb\p@<\@tempa\p@\let\@tempa\@tempb\fi% select the smaller of both
  \ifdim\@tempa\p@<\p@\scalebox{\@tempa}{\usebox\pandoc@box}%
  \else\usebox{\pandoc@box}%
  \fi%
}
% Set default figure placement to htbp
\def\fps@figure{htbp}
\makeatother
\setlength{\emergencystretch}{3em} % prevent overfull lines
\providecommand{\tightlist}{%
  \setlength{\itemsep}{0pt}\setlength{\parskip}{0pt}}
\setcounter{secnumdepth}{5}
\usepackage{booktabs}

\usepackage{color}
\usepackage{framed}
\setlength{\fboxsep}{.8em}

% These colours were manually entered, they shouldn't matter unless you want pdf output

\newenvironment{redbox}{
  \definecolor{shadecolor}{RGB}{243, 154, 157}
  \color{white}
  \begin{shaded}}
 {\end{shaded}}

\newenvironment{bluebox}{
  \definecolor{shadecolor}{RGB}{172, 210, 237}
  \color{white}
  \begin{shaded}}
 {\end{shaded}}

\newenvironment{greenbox}{
  \definecolor{shadecolor}{RGB}{141, 181, 128}
  \color{white}
  \begin{shaded}}
 {\end{shaded}}
\usepackage[]{natbib}
\bibliographystyle{plainnat}
\usepackage{bookmark}
\IfFileExists{xurl.sty}{\usepackage{xurl}}{} % add URL line breaks if available
\urlstyle{same}
\hypersetup{
  pdftitle={CBW Bookdown Guide},
  pdfauthor={Julia Qiu, Nia Hughes},
  hidelinks,
  pdfcreator={LaTeX via pandoc}}

\title{CBW Bookdown Guide}
\author{Julia Qiu, Nia Hughes}
\date{Last Updated: 2025-08-11}

\begin{document}
\maketitle

{
\setcounter{tocdepth}{1}
\tableofcontents
}
\part{Introduction}\label{part-introduction}

\chapter{CBW's Bookdown Documentation}\label{cbws-bookdown-documentation}

Welcome to CBW's documentation for creating a workshop website using Bookdown. Bookdown is an R package that is used to build books, and in our case, the websites hosting CBW's workshops! You will only need to know markdown and whatever coding language you will be using to learn bookdown.

Please note: this is the documentation to create a workshop using \emph{bookdown}. If \textbf{Jupyter Book} suits you better, see \href{https://cbw-dev.github.io/jupyterbook-docs/}{here}.

If you don't know which one to use, use the following flowchart to help decide:

\chapter{Regional Coordinator Cheat Sheet}\label{regional-coordinator-cheat-sheet}

This section will teach you to set up your GitHub Pages site for workshops. Note that it assumes you've already followed the installation instructions in the next section.

\section{Create your repo}\label{create-your-repo}

\begin{enumerate}
\def\labelenumi{\arabic{enumi}.}
\item
  At the \textbf{bookdown template}, click ``Use this template'' and then ``Create a new repository''.
\item
  Set your details
\end{enumerate}

\begin{itemize}
\tightlist
\item
  Repo name = workshop code (e.g.~INR\_Mon-2510)
\item
  Visibility = public
\item
  Save
\end{itemize}

\begin{enumerate}
\def\labelenumi{\arabic{enumi}.}
\setcounter{enumi}{2}
\tightlist
\item
  Configure your workshop
\end{enumerate}

\begin{itemize}
\tightlist
\item
  Open workshop\_config.json in your browser and replace all variables. Commit.
\end{itemize}

\begin{enumerate}
\def\labelenumi{\arabic{enumi}.}
\setcounter{enumi}{3}
\tightlist
\item
  Deploy
\end{enumerate}

\begin{itemize}
\tightlist
\item
  From your repo on GitHub, go to Settings \textgreater{} Pages
\item
  Ensure it is set to ``Deploy from a branch''
\item
  Set branch to ``main''
\item
  Change folder to /docs
\item
  Save!
\item
  Check deploy by going to Actions
\end{itemize}

\begin{enumerate}
\def\labelenumi{\arabic{enumi}.}
\setcounter{enumi}{4}
\tightlist
\item
  Clone to your computer
\end{enumerate}

\begin{itemize}
\tightlist
\item
  On your repo page on GitHub's website, click the green Code button, ensure SSH is selected, and copy the text in the box
\item
  On your local machine, open your terminal and navigate to the folder where you store your CBW Github repos
\item
  Run \texttt{git\ clone\ {[}THE\ TEXT\ YOU\ JUST\ COPIED{]}}
\end{itemize}

\begin{enumerate}
\def\labelenumi{\arabic{enumi}.}
\setcounter{enumi}{5}
\tightlist
\item
  Configure your repo
\end{enumerate}

\begin{itemize}
\tightlist
\item
  Open the .Rproj file in the folder you've just downloaded
\item
  Upload your workshop icon to the \texttt{img/} folder
\item
  In \_output.yml, replace \texttt{missingimg.png} with your icon's filename
\item
  Hit ``Build Book'' and check preview once complete
\item
  Commit and push
\end{itemize}

\section{Give your team access}\label{give-your-team-access}

\begin{enumerate}
\def\labelenumi{\arabic{enumi}.}
\tightlist
\item
  Set up your faculty team
\end{enumerate}

\begin{itemize}
\tightlist
\item
  On GitHub's website, go to the \href{https://github.com/orgs/bioinformaticsdotca/teams}{bioinformaticsdotca organization teams page}
\item
  Create a team named as your workshop code (e.g.~INR\_Mon-2510); leave all settings as their defaults
\item
  Add your instructors and TAs to the team
\end{itemize}

\begin{enumerate}
\def\labelenumi{\arabic{enumi}.}
\setcounter{enumi}{1}
\tightlist
\item
  Give repo access
\end{enumerate}

\begin{itemize}
\tightlist
\item
  On your repo page on GitHub's website, navigate to Settings \textgreater{} Collaborators and Teams
\item
  Click ``Add Teams'' and select your team
\item
  Give your team Maintain access
\end{itemize}

\chapter{Instructor Crash Course}\label{instructor-crash-course}

\section{Clone your repo}\label{clone-your-repo}

Your coordinator will have already created your repository.

\chapter{Installations}\label{installations}

  \bibliography{book.bib,packages.bib}

\end{document}
